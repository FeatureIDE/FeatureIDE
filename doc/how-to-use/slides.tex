\documentclass[hyperref={pdfpagelabels=false}]{beamer}
% [hyperref={pdfpagelabels=false}] prevent from warning
% [notes] Notizen mit ausgeben
% [notes=only] nur Notizen ausgeben
% [handout] Overlays ignorieren

\usepackage{eurosym} % Eurozeichen: \euro
%\usepackage{subfigure} % Geteilte Abbildungen (subfig geht nicht bei beamer)
%\usepackage[german]{babel} % Unterst�tzung der deutschen Sprache
\usepackage[latin1]{inputenc} % Umlaute k�nnen direkt eingegeben werden
%\usepackage{graphicx} % Einbindung von Graphiken
\usepackage{booktabs} % toprule, midrule, bottomrule
\usepackage{calc} % Berechnungen von H�hen und Breiten
%\usepackage{animate}

% choose one of the following to prevent from font warnings
  %\let\Tiny=\tiny
  \usepackage{lmodern}
% print notes on the same page
  %\usepackage{pgfpages}
  %\setbeameroption{show notes on second screen=right}

% %%%%%%%%%%%%%%%%%%%%%%%%%%%%%%%%%%%%%%%%%%%%%%%%%%%%%%%%%%%%%%%%%%%%%%%%% %
% Ausgelagerte Makros und Stile                                             %
% %%%%%%%%%%%%%%%%%%%%%%%%%%%%%%%%%%%%%%%%%%%%%%%%%%%%%%%%%%%%%%%%%%%%%%%%% %

% Einstellung von Standardthemes
%\usepackage{beamerthemesplit}
%\usepackage{beamerthemeshadow}
\useinnertheme{default}
% default, circles, rectangels, rounded, inmargin
\useoutertheme{default}
% default, infolines, miniframe, smoothbars, sidebar, split, shadow, tree, smoothtree
\usecolortheme{dove}
% gut: beaver default dolphin dove seagull seahorse
% schlecht: albatross beetle crane fly sidebartab structure whale wolverine

% Eigener Hintergrund
\newcommand{\slidebackground}[2]{
	\tikz[remember picture,overlay] \node[text height=#1,text width=\paperwidth,below,top color=blue!25,bottom color=white] at (current page.north) {};
	\tikz[remember picture,overlay] \node[text height=#2,text width=\paperwidth,above,fill=blue!25,path fading=north,yshift=-.1mm] at (current page.south) {};
}
\newcommand{\ottokopf}{\includegraphics[width=5.5cm,trim=1.5cm 1.5cm 1.5cm 1.5cm]{../ttslides/ovgu_logo}
}
\newcommand{\setslidebackground}{
	\usebackgroundtemplate{
		\parbox[b][\paperheight+1.25cm]{\paperwidth+1.25cm}{
			\hfill\tikz 
			\node {\ottokopf}
				node[fill=white,path fading=south,fading angle=45] {\phantom{\ottokopf}}
				node[fill=white,opacity=0.66] {\phantom{\ottokopf}}
			;
		}
	}
}

% Anpassung von Kopfzeile und Hintergrund
\setbeamertemplate{navigation symbols}{}
\setbeamertemplate{frametitle}{
  \slidebackground{10ex}{5ex}
  \usebeamercolor[fg]{frametitle}
  \leavevmode
  \color{black}{\hrule}
  \hbox{
    \begin{beamercolorbox}[wd=.98\textwidth,ht=2.5ex,dp=1.125ex,leftskip=-.5cm,rightskip=-.5cm]{author in head/foot_}%
			%\includegraphics[height=0.8cm]{logo-iti}
	  	\hfill
      \insertframetitle
	  	\hfill ~
			%\includegraphics[height=1cm]{ovgu}
    \end{beamercolorbox}
  }
  \color{black}{\hrule}
}

% Anpassung der Fu�zeile
\setbeamertemplate{footline}{
  \normalfont 
  \leavevmode
  \hbox{
		\begin{beamercolorbox}[wd=.98\paperwidth,ht=2.5ex,dp=2.125ex,leftskip=.3cm,rightskip=.3cm]{title in head/foot_}
			\insertauthor
			\hfill 
			\insertshorttitle
			\hfill
			\insertframenumber
    \end{beamercolorbox}
	}
}

% Email-Adressen
\newcommand{\email}[1]{
	\href{mailto:#1}{#1}
}

% Aussagenlogische Ausdr�cke
\newcommand{\pand}{\wedge}
\newcommand{\por}{\vee}
\newcommand{\pnot}{\neg}
\newcommand{\pequals}{\Leftrightarrow}
\newcommand{\pimplies}{\Rightarrow}
\newcommand{\pnimplies}{\nRightarrow}
\newcommand{\patmostone}{\mbox{\textit{atmost1}}}
\newcommand{\pchooseone}{\mbox{\textit{choose1}}}
\newcommand{\true}{\texttt{true}}
\newcommand{\false}{\texttt{false}}

% TODOs
\newcommand{\todo}[1]{
	\textbf{\color{red} TODO: #1}
}

\newcommand{\tttitlepage}{
	\graphicspath{{pics/}}

	\section{Title Page}
	\frame{
		\thispagestyle{empty}
		\slidebackground{20ex}{15ex}
		\begin{center}
			\includegraphics[width=\textwidth]{../ttslides/INF_SIGN_druck}\\
			\vspace{.5cm}
			{\footnotesize~\\}
			{\LARGE \inserttitle\\}
			\vspace{.7cm}
			\insertauthor\\
			\vspace{.2cm}
			{\footnotesize\insertdate\\}
			\vspace{.7cm}
		\end{center}
	}

	\setslidebackground
}
% Color Scheme http://colorschemedesigner.com/#3w40I--ALK-K-
% Base Color of the OVGU INF logo, tetraed, -45�
\definecolor{blue1}{RGB}{0,105,180}%95
\definecolor{blue2}{RGB}{40,87,121}
\definecolor{blue3}{RGB}{0,57,97}
\definecolor{blue4}{RGB}{76,166,230}
\definecolor{blue5}{RGB}{136,191,230}
\definecolor{orange1}{RGB}{255,144,0}%133
\definecolor{orange2}{RGB}{171,121,56}
\definecolor{orange3}{RGB}{137,78,0}
\definecolor{orange4}{RGB}{255,181,84}
\definecolor{orange5}{RGB}{255,210,151}
\definecolor{green1}{RGB}{11,215,0}%75
\definecolor{green2}{RGB}{52,144,48}
\definecolor{green3}{RGB}{6,116,0}
\definecolor{green4}{RGB}{88,241,80}
\definecolor{green5}{RGB}{148,241,143}
\definecolor{red1}{RGB}{253,0,6}%86
\definecolor{red2}{RGB}{170,56,59}
\definecolor{red3}{RGB}{136,0,3}
\definecolor{red4}{RGB}{254,84,88}
\definecolor{red5}{RGB}{254,151,154}

\definecolor{background}{named}{white}
\definecolor{bgborder}{named}{black}
\definecolor{comment}{named}{red3}

\definecolor{blue}{named}{blue1}
\definecolor{green}{named}{green1}
\definecolor{red}{named}{red1}
\definecolor{orange}{named}{orange1}

%\definecolor{coqred}{named}{red1}
%\definecolor{coqblue}{named}{blue1}
%\definecolor{coqcomment}{named}{comment}
\definecolor{coqred}{named}{black}
\definecolor{coqblue}{named}{black}
\definecolor{coqcomment}{named}{black}

\definecolor{acolor1}{named}{red1}
\definecolor{acolor2}{named}{blue1}
\definecolor{gacolor1}{gray}{0.34}%86
\definecolor{gacolor2}{gray}{0.37}%95
\definecolor{highlight}{gray}{0.9}

\definecolor{pdflinkcolor}{named}{blue3}
\definecolor{pdfcitecolor}{named}{green3}

\usepackage{pgfplots}
\usepackage{tikz}
	\usetikzlibrary{arrows,positioning,backgrounds,fit,trees} 
	\usetikzlibrary{fadings,shapes.geometric}%spy,
	\usetikzlibrary{decorations,scopes,calc,decorations.pathreplacing}



\title{How to Use FeatureIDE}
\author{Thomas Th�m}
\date{\today}

\begin{document}

\tttitlepage

\subsection{Content}
\begin{frame}%[fragile]
	\frametitle{\insertsubsection}
	\begin{itemize}
		\item What is Feature-Oriented Software Development?\only<1>{\\[5mm]}
		\only<2>{
			\begin{itemize}
				\item Feature-Oriented Programming + Example
				\item Configurations
				\item Feature Model
				\item Composition Engines\\[5mm]
			\end{itemize}
			\color{grayed}
		}
		\item What functionality does FeatureIDE provide?\\[5mm]
		\item How to start working with FeatureIDE?
	\end{itemize}
\end{frame}

\section{FOSD Background}

\subsection{Feature-Oriented Programming (FOP)}
\begin{frame}%[fragile]
	\frametitle{\insertsubsection}
	\begin{itemize}
	  \item Introduced 1997 by Christian Prehofer
		\item Based on Object-Oriented Programming
		\item Features realize functionalities
		\item Features are cross-cutting to objects
		\item Features modularize fragments from certain classes
		\item Fragment contains some methods/fields of a class belonging to one functionality
		\item Goals: code traceability, software customization
	\end{itemize}
\end{frame}

\subsection{FOP Example}
\begin{frame}%[fragile]
	\frametitle{\insertsubsection}
	\begin{center}
		\includegraphics[width=\textwidth]{FOPexample}\\
		\tiny\url{http://wwwiti.cs.uni-magdeburg.de/iti_db/lehre/epmd/2009/slides/06_FOP.pdf}
	\end{center}
\end{frame}

\subsection{Configuration}
\begin{frame}%[fragile]
	\frametitle{\insertsubsection}
	\begin{center}
		\includegraphics[scale=.75]{conf}
	\end{center}
	\begin{itemize}
		\item Selection of features
		\item Composition of features results in a program variant
		\item Not all combinations are useful
	\end{itemize}
\end{frame}

\subsection{Feature Model}
\begin{frame}%[fragile]
	\frametitle{\insertsubsection}
	\begin{itemize}
		\item Specifies valid combinations of features
		\item Graphically represented by a feature diagram
		\item Created for a particular domain
		\item Describes a software product line (SPL)
	\end{itemize}
	\begin{center}
    \begin{tikzpicture}
      \node at (0,0) {\includegraphics[scale=0.75]{tinygpl1}};
      \node[overlay,inner sep=.5mm,shape=rectangle,draw] at (4.5,-0.9) {
				\begin{minipage}{21mm}
					\tiny
					\begin{flushleft}
						\includegraphics[trim=12.5mm 10mm 14mm 5.5mm,clip,scale=0.6]{fm_and}\ And-group\\[.5mm]
						\includegraphics[trim=12.5mm 10mm 14mm 5.5mm,clip,scale=0.6]{fm_or}\ Or-group\\[.5mm]
						\includegraphics[trim=12.5mm 10mm 14mm 5.5mm,clip,scale=0.6]{fm_alt}\ Alternative-group\\[.5mm]
						\includegraphics[trim=6mm 4.5mm 21mm 11mm,clip,scale=0.6]{fm_and}\ Mandatory\\[.5mm]
						\includegraphics[trim=19.5mm 4.5mm 7.5mm 11mm,clip,scale=0.6]{fm_and}\ Optional
					\end{flushleft}
				\end{minipage}
			};
      \uncover<2>{
	      \node[overlay] (label) at (-2.0,1.2) {\color{red}\footnotesize And-group};
	      \draw[red,thick,-latex,overlay] (label) -- (-0.2,0.8);
			}      
      \uncover<3>{
	      \node[overlay] (label) at (-2.9,0.6) {\color{red}\footnotesize Mandatory};
	      \draw[red,thick,-latex,overlay] (label) -- (-1.1,0.45);
			}      
      \uncover<4>{
	      \node[overlay] (label) at (2.9,1.0) {\color{red}\footnotesize Optional};
	      \draw[red,thick,-latex,overlay] (label) -- (1.4,0.45);
			}      
      \uncover<5>{
	      \node[overlay] (label) at (-3.4,-0.4) {\color{red}\footnotesize Alternative-group};
	      \draw[red,thick,-latex,overlay] (label) -- (-1.2,-0.1);
			}      
      \uncover<6>{
	      \node[overlay] (label) at (3,0.1) {\color{red}\footnotesize Or-group};
	      \draw[red,thick,-latex,overlay] (label) -- (1.4,-0.1);
			}      
      \uncover<7>{
	      \node[overlay] (label) at (-3.1,-1.4) {\color{red}\footnotesize Cross-tree constraints};
	      \draw[red,thick,-latex,overlay] (label) -- (-0.6,-1.3);
			}
			\only<8>{}
    \end{tikzpicture}
	\end{center}
\end{frame}

\subsection{Composition Engines}
\begin{frame}%[fragile]
	\frametitle{\insertsubsection}
	Command-line tools used to compose files within FeatureIDE:\\[5mm]
	\begin{itemize}
		\item AHEAD (jampack): .jak (Java 1.4)\\
		  \url{http://userweb.cs.utexas.edu/~schwartz/ATS.html}\\[5mm]
		\item FeatureC++: .cpp (C++)\\
		  \url{http://www.fosd.de/fcpp}\\[5mm]
		\item FeatureHouse: .java (Java 1.5), .cs (C\#), .c/.h (C), .hs (Haskell), .jj (JavaCC), .als (Alloy), .xmi (UML)\\
		  \url{http://www.fosd.de/fh}		
	\end{itemize}
\end{frame}

\newcommand{\picfontsize}{\footnotesize}
\newcommand{\picfm}{\picfontsize\includegraphics[width=.8\textwidth]{tinygpl1}\\Feature Model}
\newcommand{\piccodebase}{\picfontsize\includegraphics[width=.5\textwidth]{codebase1}\includegraphics[width=.5\textwidth]{codebase2}\\Feature Modules}
\newcommand{\picconf}{\picfontsize\includegraphics[width=.85\textwidth]{conf}\\Configurations}
\newcommand{\picgen}{\picfontsize\includegraphics[width=.5\textwidth]{blackbox}\\Composer}
\newcommand{\picvariant}{\picfontsize\includegraphics[width=.75\textwidth]{programvariant}\\Program Variants}
\newcommand{\mnode}[3]{
	\node[xshift=.3em, yshift=.3em, #1] (#2x) {};
	\node[#1] {};
	\node[xshift=-.3em, yshift=-.3em, #1] (#2) {#3}
}
\subsection{Feature-Oriented Software Development}
\begin{frame}%[fragile]
	\frametitle{\insertsubsection}
	\begin{center}
		\begin{tikzpicture}[node distance=8mm]
			\node[node1s] (fm) {\picfm};
			\node[node1s, right=of fm]	(codebase) {\piccodebase};
			\mnode{node1s, below=of fm}{configuration}{\picconf};
			\node[node1s, below=of codebase] (generator) {\picgen};
			\mnode{node1s, right=of generator}{variants}{\picvariant};
			\draw[edge2] (fm) -- (configurationx);
			%\draw[edge2] (fm) -- (codebase);
			\draw[edge1] (configurationx) -- (generator);
			\draw[edge1] (codebase) -- (generator);
			\draw[edge1] (generator) -- (variants);
		\end{tikzpicture}
	\end{center}
\end{frame}

\begin{frame}%[fragile]
	\frametitle{Content}
	\begin{itemize}
		\item {\color{grayed}What is Feature-Oriented Software Development?}\\[5mm]
		\item What functionality does FeatureIDE provide?
			\begin{itemize}
				\item Feature Model Editor + Edit View
				\item Configuration Editor
				\item Jak Editor
				\item Collaboration Diagram
				\item Feature Project Builder
				\item Run Configurations
				\item Creation Wizards\\[5mm]
			\end{itemize}
		\item {\color{grayed}How to start working with FeatureIDE?}
	\end{itemize}
\end{frame}

\section{Functionality of FeatureIDE}

\subsection{Feature Model Editor 1/4}
\begin{frame}%[fragile]
	\frametitle{\insertsubsection}
	\begin{itemize}
		\item \ldots
		\item \ldots
		\item \ldots
	\end{itemize}
\end{frame}

\subsection{Feature Model Editor 2/4}
\begin{frame}%[fragile]
	\frametitle{\insertsubsection}
	\begin{itemize}
		\item \ldots
		\item \ldots
		\item \ldots
	\end{itemize}
\end{frame}

\subsection{Feature Model Editor 3/4}
\begin{frame}%[fragile]
	\frametitle{\insertsubsection}
	\begin{itemize}
		\item \ldots
		\item \ldots
		\item \ldots
	\end{itemize}
\end{frame}

\subsection{Feature Model Editor 4/4}
\begin{frame}%[fragile]
	\frametitle{\insertsubsection}
	\begin{itemize}
		\item \ldots
		\item \ldots
		\item \ldots
	\end{itemize}
\end{frame}

\subsection{Feature Model Edit View}
\begin{frame}%[fragile]
	\frametitle{\insertsubsection}
	\begin{itemize}
		\item \ldots
		\item \ldots
		\item \ldots
	\end{itemize}
\end{frame}

\subsection{Configuration Editor}
\begin{frame}%[fragile]
	\frametitle{\insertsubsection}
	\begin{itemize}
		\item \ldots
		\item \ldots
		\item \ldots
	\end{itemize}
\end{frame}

\subsection{Jak Editor}
\begin{frame}%[fragile]
	\frametitle{\insertsubsection}
	\begin{itemize}
		\item \ldots
		\item \ldots
		\item \ldots
	\end{itemize}
\end{frame}

\subsection{Collaboration Diagram}
\begin{frame}%[fragile]
	\frametitle{\insertsubsection}
	\begin{itemize}
		\item \ldots
		\item \ldots
		\item \ldots
	\end{itemize}
\end{frame}

\subsection{Feature Project Builder}
\begin{frame}%[fragile]
	\frametitle{\insertsubsection}
	\begin{itemize}
		\item \ldots
		\item \ldots
		\item \ldots
	\end{itemize}
\end{frame}

\subsection{Run Configurations}
\begin{frame}%[fragile]
	\frametitle{\insertsubsection}
	\begin{itemize}
		\item \ldots
		\item \ldots
		\item \ldots
	\end{itemize}
\end{frame}

\subsection{Creation Wizards}
\begin{frame}%[fragile]
	\frametitle{\insertsubsection}
	\begin{itemize}
		\item \ldots
		\item \ldots
		\item \ldots
	\end{itemize}
\end{frame}

\begin{frame}%[fragile]
	\frametitle{Content}
	\begin{itemize}
		\item {\color{grayed}What is Feature-Oriented Software Development?\\[5mm]
		\item What functionality does FeatureIDE provide?}\\[5mm]
		\item How to start working with FeatureIDE?
		\begin{itemize}
			\item FeatureIDE Installation
			\item Feature Project Structure
			\item Cheat Sheet
		\end{itemize}
	\end{itemize}
\end{frame}

\section{How to Start}

\subsection{Installation of FeatureIDE}
\begin{frame}%[fragile]
	\frametitle{\insertsubsection}
	\begin{itemize}
		\item \ldots
		\item \ldots
		\item \ldots
	\end{itemize}
\end{frame}

\subsection{Feature Project Structure}
\begin{frame}%[fragile]
	\frametitle{\insertsubsection}
	\begin{itemize}
		\item \ldots
		\item \ldots
		\item \ldots
	\end{itemize}
\end{frame}

\subsection{Cheat Sheet}
\begin{frame}%[fragile]
	\frametitle{\insertsubsection}
	\begin{itemize}
		\item \ldots
		\item \ldots
		\item \ldots
	\end{itemize}
\end{frame}

%\subsection{\ldots}
%\begin{frame}%[fragile]
	%\frametitle{\insertsubsection}
	%\begin{itemize}
		%\item \ldots
		%\item \ldots
		%\item \ldots
	%\end{itemize}
%\end{frame}

\end{document}